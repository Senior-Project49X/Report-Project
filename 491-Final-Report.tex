\documentclass[semifinal]{cpecmu}


%% This is a sample document demonstrating how to use the CPECMU
%% project template. If you are having trouble, see "cpecmu.pdf" for
%% documentation.

\projectNo{P069-1}
\acadyear{2021}

\titleTH{โครงงานสุดเลิฟของฉัน}
\titleEN{Your Project Name Goes Here}

\author{นายกินรี ไทร์ล้ำเลิศ}{Kinnaree Tirelumlert}{690610696}
\author{นายบรรจบ พบเอฟตลอด}{Banjob Pob-eftalord}{690610969}

\cpeadvisor{chinawat}
\cpecommittee{paskorn}
\committee{รศ.ดร.\,นิพนธ์ ธีรอำพน}{Assoc.\,Prof.\,Nipon Theera-Umpon, Ph.D.}

%% Some possible packages to include:
\usepackage[final]{graphicx} % for including graphics

%% Add bookmarks and hyperlinks in the document.
\PassOptionsToPackage{hyphens}{url}
\usepackage[colorlinks=true,allcolors=Blue4,citecolor=red,linktoc=all]{hyperref}
\def\UrlLeft#1\UrlRight{$#1$}

%% Needed just by this example, but maybe not by most reports
\usepackage{afterpage} % for outputting
\usepackage{pdflscape} % for landscape figures and tables. 

%% Some other useful packages. Look these up to find out how to use
%% them.
% \usepackage{natbib}    % for author-year citation styles
% \usepackage{txfonts}
% \usepackage{appendix}  % for appendices on a per-chapter basis
% \usepackage{xtab}      % for tables that go over multiple pages
% \usepackage{subfigure} % for subfigures within a figure
% \usepackage{pstricks,pdftricks} % for access to special PostScript and PDF commands
% \usepackage{nomencl}   % if you have a list of abbreviations

%% if you're having problems with overfull boxes, you may need to increase
%% the tolerance to 9999
% \tolerance=9999

\bibliographystyle{plain}
% \bibliographystyle{IEEEbib}

% \renewcommand{\topfraction}{0.85}
% \renewcommand{\textfraction}{0.1}
% \renewcommand{\floatpagefraction}{0.75}

%% Example for glossary entry
%% Need to use glossary option
%% See glossaries package for complete documentation.
\ifglossary
  \newglossaryentry{lorem ipsum}{
    name=lorem ipsum,
    description={derived from Latin dolorem ipsum, translated as ``pain itself''}
  }
\fi

%% Uncomment this command to preview only specified LaTeX file(s)
%% imported with \include command below.
%% Any other file imported via \include but not specified here will not
%% be previewed.
%% Useful if your report is large, as you might not want to build
%% the entire file when editing a certain part of your report.
% \includeonly{chapters/intro,chapters/background}

\begin{document}
\maketitle
% \makesignature

\ifproject
\begin{abstractTH}
เขียนบทคัดย่อของโครงงานที่นี่

การเขียนรายงานเป็นส่วนหนึ่งของการทำโครงงานวิศวกรรมคอมพิวเตอร์
เพื่อทบทวนทฤษฎีที่เกี่ยวข้อง อธิบายขั้นตอนวิธีแก้ปัญหาเชิงวิศวกรรม และวิเคราะห์และสรุปผลการทดลองอุปกรณ์และระบบต่างๆ
\enskip อย่างไรก็ดี การสร้างรูปเล่มรายงานให้ถูกรูปแบบนั้นเป็นขั้นตอนที่ยุ่งยาก
แม้ว่าจะมีต้นแบบสำหรับใช้ในโปรแกรม Microsoft Word แล้วก็ตาม
แต่นักศึกษาส่วนใหญ่ยังคงค้นพบว่าการใช้งานมีความซับซ้อน และเกิดความผิดพลาดในการจัดรูปแบบ กำหนดเลขหัวข้อ และสร้างสารบัญอยู่
\enskip ภาควิชาวิศวกรรมคอมพิวเตอร์จึงได้จัดทำต้นแบบรูปเล่มรายงานโดยใช้ระบบจัดเตรียมเอกสาร
\LaTeX{} เพื่อช่วยให้นักศึกษาเขียนรายงานได้อย่างสะดวกและรวดเร็วมากยิ่งขึ้น
\end{abstractTH}

\begin{abstract}
The abstract would be placed here. It usually does not exceed 350 words
long (not counting the heading), and must not take up more than one (1) page
(even if fewer than 350 words long).

Make sure your abstract sits inside the \texttt{abstract} environment.
\end{abstract}

\iffalse
\begin{dedication}
This document is dedicated to all Chiang Mai University students.

Dedication page is optional.
\end{dedication}
\fi % \iffalse

\begin{acknowledgments}
Your acknowledgments go here. Make sure it sits inside the
\texttt{acknowledgment} environment.

\acksign{2020}{5}{25}
\end{acknowledgments}%
\fi % \ifproject

\contentspage

\ifproject
\figurelistpage

\tablelistpage
\fi % \ifproject

% \abbrlist % this page is optional

% \symlist % this page is optional

% \preface % this section is optional


\pagestyle{empty}\cleardoublepage
\normalspacing \setcounter{page}{1} \pagenumbering{arabic} \pagestyle{cpecmu}

\chapter{\ifenglish Introduction\else บทนำ\fi}

\section{\ifenglish Project rationale\else ที่มาของโครงงาน\fi}
มีบุคลากรภายในมหาวิทยาลัยเชียงใหม่หลายคนต้องการที่จะหาคนมาเข้าร่วมงานอีเว้นท์ หรือ เข้ามาช่วยในงานโครงงานโปรเจคต่างๆ เเต่ไม่สามารถหาผู้เข้าร่วมได้ ซึ่งในหลายๆครั้งนั้นอาจมีนักศึกษาหรือบุคลากรจำนวนมากที่สนใจเข้าร่วมเเต่ไม่ได้เข้าร่วม เพียงเพราะไม่ได้รับข่าวสารการประกาศ ซึ่งอาจเป็นเพราะด้วยช่องทางที่ผู้จัดประกาศนั้นเข้าไม่ถึงบุคลากรเหล่านั้นด้วยเหตุผลต่างๆอาทิเช่น ประกาศในโซเชียลมีเดียเเล้วผู้ที่สนใจไม่เห็นเนื่องด้วยอาจไม่ได้ติดตามช่องทางที่ประกาศหรืออาจเพราะถูกบดบังด้วยอัลกอริทึมของโซเชียลมีเดียนั้นๆ


\section{\ifenglish Objectives\else วัตถุประสงค์ของโครงงาน\fi}
\begin{enumerate}
    \item พัฒนาเว็บที่สามารถรองรับการประกาศกิจกรรม เเละหัวข้อ senior projectต่างๆ ให้ผู้ที่สนใจนั้นสามารถเข้ามามีส่วนร่วมกันได้อย่างมีประสิทธิภาพ
    \item พัฒนาเว็บที่นำความรู้ทางด้านData Analytic มาใช้ในการเเนะนำกิจกรรม หรือหัวข้อ senior project ให้เเก่ผู้ใช้โดยกิจกรรมที่เเนะนำจะต้องสอดคล้องกับความสนใจของผู้ใช้รายนั้นๆ
\end{enumerate}

\section{\ifenglish Project scope\else ขอบเขตของโครงงาน\fi}

\subsection{\ifenglish Hardware scope\else ขอบเขตด้านฮาร์ดแวร์\fi}
\begin{enumerate}
    \item คอมพิวเตอร์เพื่อใช้พัฒนาเว็บแอปพลิเคชันและตรวจสอบผลลัพธ์ผ่านเว็บบราวเซอร์
    \item สมาร์ทโฟนระบบแอนดรอยด์เพื่อใช้ตรวจสอบผลลัพธ์ผ่านเว็บบราวเซอร์
\end{enumerate}
\subsection{\ifenglish Software scope\else ขอบเขตด้านซอฟต์แวร์\fi}
\begin{enumerate}
    \item การเข้าถึงเว็บแอพลิเคชัน สามารถเข้าผ่านเว็บบราวเซอร์ต่างๆ เช่น Chrome, Firefox เป็นต้น
    \item ส่วนบัญชีผู้ใช้ คือการยืนยันตัวตนผ่าน CMU-OAuth
    \item ส่วนแสดงกิจกรรมทั้งหมด ผู้ใช้จะสามารถดูข้อมูลเบื้องต้นของกิจกรรมต่างๆได้ โดยในส่วนนี้จะแบ่งเป็นกิจกรรมที่มีผู้สนใจเยอะ และกิจกรรมทั้งหมด
    \item ส่วนการสร้างกิจกรรม ผู้ใช้สามารถสร้างกิจกรรมใหม่ขึ้นมา โดยระบุรายละเอียดต่างๆของกิจกรรม สามารถเลือกได้ว่าจะต้องการผู้สมัครเข้าร่วมกิจกรรมหรือไม่
    \item ส่วนแสดงกิจกรรมเฉพาะ เมื่อผู้ใช้เข้ามาส่วนนี้ ผู้ใช้จะสามารถดูข้อมูลของกิจกรรมได้โดยละเอียด และสามารถสมัครเข้าร่วมกิจกรรมได้
    \item ส่วนตอบรับการเข้าร่วมกิจกรรม ผู้ที่สร้างกิจกรรมสามารถเลือกได้ว่าจะให้ผู้สมัครคนไหนมีสิทธิเข้าร่วมกิจกรรมบ้าง
    \item ส่วนการให้คะแนนกิจกรรม ผู้เข้าร่วมกิจกรรมสามารถให้คะแนนกิจกรรมและผู้จัดได้ ส่วนผู้จัดก็สามารถให้คะแนนผู้เข้าร่วมได้เช่นกัน
    \item ส่วนแดชบอร์ด ผู้ใช้สามารถดูสถิติต่างๆที่ตนเองสนใจได้ เช่น ผู้สร้างกิจกรรมสามารถดูผลตอบรับของผู้ใช้คนอื่นๆ, ผู้ดูแลระบบสามารถดูสถิติโดยรวมของเว็บแอพลิเคชันได้ เป็นต้น
\end{enumerate}

\subsection{ขอบเขตด้านกลุ่มผู้ใช้}
นักศึกษาและบุคลากรของมหาวิทยาลัยเชียงใหม่ที่มี CMU-Account

\subsection{ขอบเขตด้านข้อมูล}
\begin{enumerate}
    \item กิจกรรมประเภทต่างๆ เช่น รับน้องขึ้นดอย, CPE Music box, จับกลุ่มออกกำลังกาย เป็นต้น
    \item ข้อมูลของผู้ใช้ที่ได้รับจาก CMU-Account
\end{enumerate}

\section{\ifenglish Expected outcomes\else ประโยชน์ที่ได้รับ\fi}
\begin{enumerate}
    \item สามารถทำให้กิจกรรมต่างๆที่มาฝากประกาศในช่องทางเรา เข้าถึงกลุ่มเป้าหมายได้มากขึ้น
    \item สามารถทำให้นักศึกษาและบุคลากรในมหาวิทยาลัย ได้เห็นกิจกรรมที่ตัวเองสนใจได้ง่ายขึ้น
    \item สามารถทำให้การหาข้อมูลกิจกรรมต่างๆนั้น สะดวกมากยิ่งขึ้น
\end{enumerate}

\section{\ifenglish Technology and tools\else เทคโนโลยีและเครื่องมือที่ใช้\fi}

\subsection{\ifenglish Hardware technology\else เทคโนโลยีด้านฮาร์ดแวร์\fi}
\begin{enumerate}
    \item ASUS Vivobook Pro 15 : สำหรับพัฒนาเว็บแอพลิเคชัน
    \item Huawei P20 Pro : สำหรับตรวจสอบการแสดงผลบนสมาร์ทโฟน
\end{enumerate}
\subsection{\ifenglish Software technology\else เทคโนโลยีด้านซอฟต์แวร์\fi}
\begin{enumerate}
    \item Figma : เว็บแอพลิเคชันที่ใช้ในการออกแบบ Prototype ของเว็บไซต์
    \item Jira Software : เว็บแอพลิเคชันที่ใช้ในการวางแผนงาน, แบ่งงาน และดูความคืบหน้าของแต่ละงาน
    \item GitHub : Version control ที่สามารถเก็บไฟล์ได้บนอินเทอร์เน็ต
    \item Visual Studio Code : Text Editor ที่ใช้ในการเขียนโค้ด โดยมีจุดเด่นคือมีส่วนขยายโปรแกรมที่สร้างโดยผู้ใช้ทั่วโลก
    \item React : Javascript Library ที่ช่วยในการสร้าง User interface
    \item TypeScript : ภาษาโปรแกรมที่พัฒนาต่อมาจาก Javascript โดยเพิ่ม Static typing เพื่อตรวจสอบความผิดพลาดของโปรแกรมได้โดยง่าย
    \item MySQL : ฐานข้อมูล
    \item Firebase : แพลตฟอร์มที่ใช้พัฒนา backend และจัดการฐานข้อมูลผ่านเว็บบราวเซอร์
\end{enumerate}
\section{\ifenglish Project plan\else แผนการดำเนินงาน\fi}

\begin{plan}{6}{2023}{3}{2024}
    \planitem{6}{2023}{6}{2023}{ค้นหาหัวข้อที่สนใจและอาจารย์ที่ปรึกษา}
    \planitem{6}{2023}{8}{2023}{ค้นหาข้อมูล ทฤษฎีที่เกี่ยวข้องและกำหนดขอบเขต}
    \planitem{7}{2023}{8}{2023}{ออกแบบ Mockup คร่าวๆของเว็บด้วย Figma}
    \planitem{8}{2023}{8}{2023}{ออกแบบ Diagram ของระบบแบบคร่าวๆ}
    \planitem{8}{2023}{9}{2023}{หาข้อมูลเกี่ยวกับกิจกรรมตัวอย่าง}
    \planitem{8}{2023}{10}{2023}{ออกแบบ Flow ของระบบ}
    \planitem{8}{2023}{10}{2023}{ออกแบบ UX/UI ของเว็บด้วย Figma}
    \planitem{9}{2023}{10}{2023}{เขียนรายงานและนำเสนอ 261491}
    \planitem{10}{2023}{10}{2023}{ศึกษา Algorithm สำหรับระบบ Recommendation}
    \planitem{10}{2023}{10}{2023}{ศึกษาการทำ Data Visualization สำหรับหน้าแดชบอร์ด}
    \planitem{10}{2023}{11}{2023}{ออกแบบฐานข้อมูล}
    \planitem{11}{2023}{2}{2024}{พัฒนาเว็บแอพลิเคชัน}
    \planitem{11}{2023}{2}{2024}{ทดสอบกับผู้ใช้จริงและปรับปรุงระบบ}
    \planitem{1}{2024}{3}{2024}{เขียนรายงานและนำเสนอ 261492}

\end{plan}

\section{\ifenglish Roles and responsibilities\else บทบาทและความรับผิดชอบ\fi}
\begin{enumerate}
    \item ส่วนที่ทำงานร่วมกันได้แก่ การวางแผนงาน, การค้นหาความรู้และทฤษฎีที่เกี่ยวข้อง และการพัฒนาเว็บแอพลิเคชัน
    \item ส่วนที่รับผิดชอบโดยนาย ณัฏฐพล ตันจอ 620610786 ได้แก่ การออกแบบหน้าสร้างกิจกรรมและหน้าเข้าร่วมกิจกรรม, รายงานหัวข้อ(1.3, 1.5, 1.6, 2.1-2.3, 3.1-3.3) 
    \item ส่วนที่รับผิดชอบโดยนาย นายธิษณ์ธนัย แก้วเพ็ชร์ 630610741 ได้แก่ การออกแบบหน้าแรกและหน้าแสดงกิจกรรม, รายงานหัวข้อ(บทนำ, 1.1, 1.2, 1.4, 1.8, 3.4-3.5)
\end{enumerate}

\section{\ifenglish%
Impacts of this project on society, health, safety, legal, and cultural issues
\else%
ผลกระทบด้านสังคม สุขภาพ ความปลอดภัย กฎหมาย และวัฒนธรรม
\fi}

แนวทางและโยชน์ในการประยุกต์ใช้งานโครงงานกับงานในด้านอื่นๆ รวมถึงผลกระทบในด้านสังคมและสิ่งแวดล้อมจากการใช้ความรู้ทางวิศวกรรมที่ได้

\chapter{\ifenglish Background Knowledge and Theory\else ทฤษฎีที่เกี่ยวข้อง\fi}

การทำโครงงาน เริ่มต้นด้วยการศึกษาค้นคว้า ทฤษฎีที่เกี่ยวข้อง หรือ งานวิจัย/โครงงาน ที่เคยมีผู้นำเสนอไว้แล้ว ซึ่งเนื้อหาในบทนี้ก็จะเกี่ยวกับการอธิบายถึงสิ่งที่เกี่ยวข้องกับโครงงาน เพื่อให้ผู้อ่านเข้าใจเนื้อหาในบทถัดๆ ไปได้ง่ายขึ้น

\section{ด้าน Frontend (เติมรายละเอียดภายหลัง)}
\subsection{Javascript}
\subsection{React}
\subsection{TypeScript}

\section{ด้าน Backend}
\subsection{SQL Database}
\subsection{Firebase}
\subsection{CMU-OAuth}

\section{ระบบ Text PreProcessing}
คือ การแปลงข้อมูลดิบให้อยู่ในรูปแบบที่คอมพิวเตอร์เข้าใจ เช่นการตัดคำสำคัญแยกออกจากประโยค เพื่อให้สะดวกต่อการนำข้อมูลไปใช้ในการพัฒนาระบบ recommendation

\section{ระบบ Recommendation}
\subsection{Content Based Filtering}
\subsection{Collaborative Filtering}

\section{ความรู้ตามหลักสูตรที่นำมาใช้}
\subsection{ความรู้ด้านHuman Computer Interaction ใช้ในการออกแบบดีไซน์หน้าเว็บให้สื่อประสานกับกลุ่มผู้ใช้งานเป้าหมาย}
\subsection{ความรู้ด้านNatural Language Processing ใช้ประยุกต์ในการพัฒนาระบบแนะนำกิจกรรม}
\subsection{ความรู้ด้านWeb Development ใช้ในการพัฒนาเว็บไซต์}
\subsection{ความรู้ด้านDatabase ใช้ในการออกแบบฐานข้อมูล}
\subsection{ความรู้ด้านInfra and Cloud technology ใช้ในการdeployหน้าเว็บ}

\section{ความรู้นอกหลักสูตรที่นำมาใช้}
\subsection{ความรู้เรื่องการใช้Firebase ในการเก็บข้อมูล}
% \subsection{Subsection heading goes here}

% Subsection 1 text

% \subsubsection{Subsubsection 1 heading goes here}
% Subsubsection 1 text

% \subsubsection{Subsubsection 2 heading goes here}
% Subsubsection 2 text

% \section{Third section}
% Section 3 text. The dielectric constant\index{dielectric constant}
% at the air-metal interface determines
% the resonance shift\index{resonance shift} as absorption or capture occurs
% is shown in Equation~\eqref{eq:dielectric}:

% \begin{equation}\label{eq:dielectric}
% k_1=\frac{\omega}{c({1/\varepsilon_m + 1/\varepsilon_i})^{1/2}}=k_2=\frac{\omega
% \sin(\theta)\varepsilon_\mathit{air}^{1/2}}{c}
% \end{equation}

% \noindent
% where $\omega$ is the frequency of the plasmon, $c$ is the speed of
% light, $\varepsilon_m$ is the dielectric constant of the metal,
% $\varepsilon_i$ is the dielectric constant of neighboring insulator,
% and $\varepsilon_\mathit{air}$ is the dielectric constant of air.

% \section{About using figures in your report}

% % define a command that produces some filler text, the lorem ipsum.
% \newcommand{\loremipsum}{
%   \textit{Lorem ipsum dolor sit amet, consectetur adipisicing elit, sed do
%   eiusmod tempor incididunt ut labore et dolore magna aliqua. Ut enim ad
%   minim veniam, quis nostrud exercitation ullamco laboris nisi ut
%   aliquip ex ea commodo consequat. Duis aute irure dolor in
%   reprehenderit in voluptate velit esse cillum dolore eu fugiat nulla
%   pariatur. Excepteur sint occaecat cupidatat non proident, sunt in
%   culpa qui officia deserunt mollit anim id est laborum.}\par}

% \begin{figure}
%   \centering

%   \fbox{
%      \parbox{.6\textwidth}{\loremipsum}
%   }

%   % To include an image in the figure, say myimage.pdf, you could use
%   % the following code. Look up the documentation for the package
%   % graphicx for more information.
%   % \includegraphics[width=\textwidth]{myimage}

%   \caption[Sample figure]{This figure is a sample containing \gls{lorem ipsum},
%   showing you how you can include figures and glossary in your report.
%   You can specify a shorter caption that will appear in the List of Figures.}
%   \label{fig:sample-figure}
% \end{figure}

% Using \verb.\label. and \verb.\ref. commands allows us to refer to
% figures easily. If we can refer to Figures
% \ref{fig:walrus} and \ref{fig:sample-figure} by name in the {\LaTeX}
% source code, then we will not need to update the code that refers to it
% even if the placement or ordering of the figures changes.

% \loremipsum\loremipsum

% % This code demonstrates how to get a landscape table or figure. It
% % uses the package lscape to turn everything but the page number into
% % landscape orientation. Everything should be included within an
% % \afterpage{ .... } to avoid causing a page break too early.
% \afterpage{
%   \begin{landscape}
%   \begin{table}
%     \caption{Sample landscape table}
%     \label{tab:sample-table}

%     \centering

%     \begin{tabular}{c||c|c}
%         Year & A & B \\
%         \hline\hline
%         1989 & 12 & 23 \\
%         1990 & 4 & 9 \\
%         1991 & 3 & 6 \\
%     \end{tabular}
%   \end{table}
%   \end{landscape}
% }

% \loremipsum\loremipsum\loremipsum

% \section{Overfull hbox}

% When the \verb.semifinal. option is passed to the \verb.cpecmu. document class,
% any line that is longer than the line width, i.e., an overfull hbox, will be
% highlighted with a black solid rule:
% \begin{center}
% \begin{minipage}{2em}
% juxtaposition
% \end{minipage}
% \end{center}

\section{\ifenglish%
\ifcpe CPE \else ISNE \fi knowledge used, applied, or integrated in this project
\else%
ความรู้ตามหลักสูตรซึ่งถูกนำมาใช้หรือบูรณาการในโครงงาน
\fi
}

อธิบายถึงความรู้ และแนวทางการนำความรู้ต่างๆ ที่ได้เรียนตามหลักสูตร ซึ่งถูกนำมาใช้ในโครงงาน

\section{\ifenglish%
Extracurricular knowledge used, applied, or integrated in this project
\else%
ความรู้นอกหลักสูตรซึ่งถูกนำมาใช้หรือบูรณาการในโครงงาน
\fi
}

อธิบายถึงความรู้ต่างๆ ที่เรียนรู้ด้วยตนเอง และแนวทางการนำความรู้เหล่านั้นมาใช้ในโครงงาน

\chapter{\ifproject%
\ifenglish Project Structure and Methodology\else โครงสร้างและขั้นตอนการทำงาน\fi
\else%
\ifenglish Project Structure\else โครงสร้างของโครงงาน\fi
\fi
}

% ในบทนี้จะกล่าวถึงหลักการ และการออกแบบระบบ

\makeatletter

% \renewcommand\section{\@startsection {section}{1}{\z@}%
%                                    {13.5ex \@plus -1ex \@minus -.2ex}%
%                                    {2.3ex \@plus.2ex}%
%                                    {\normalfont\large\bfseries}}

\makeatother
%\vspace{2ex}
% \titleformat{\section}{\normalfont\bfseries}{\thesection}{1em}{}
% \titlespacing*{\section}{0pt}{10ex}{0pt}

\section{หลักการทำงานของแอพลิเคชัน}

% \begin{figure}
% \begin{center}
% \includegraphics{800px-Briny_Beach.jpg}
% \end{center}
% \caption[Poem]{The Walrus and the Carpenter}
% \label{fig:walrus}
% \end{figure}

\section{การใช้งานแอพลิเคชัน}
\subsection{ผู้สร้างกิจกรรม}
\subsection{ผู้เข้าร่วมกิจกรรม}
\subsection{ผู้ดูแลระบบ}
\section{นโยบายความเป็นส่วนตัว}
% \subsection{The Black Kitten}
%   One thing was certain, that the WHITE kitten had had nothing to
% do with it:---it was the black kitten's fault entirely~\cite{aiw}.  For the
% white kitten had been having its face washed by the old cat for
% the last quarter of an hour (and bearing it pretty well,
% considering); so you see that it COULDN'T have had any hand in
% the mischief.

%   The way Dinah washed her children's faces was this:  first she
% held the poor thing down by its ear with one paw, and then with
% the other paw she rubbed its face all over, the wrong way,
% beginning at the nose:  and just now, as I said, she was hard at
% work on the white kitten, which was lying quite still and trying
% to purr---no doubt feeling that it was all meant for its good.

%   But the black kitten had been finished with earlier in the
% afternoon, and so, while Alice was sitting curled up in a corner
% of the great arm-chair, half talking to herself and half asleep,
% the kitten had been having a grand game of romps with the ball of
% worsted Alice had been trying to wind up, and had been rolling it
% up and down till it had all come undone again; and there it was,
% spread over the hearth-rug, all knots and tangles, with the
% kitten running after its own tail in the middle.

% \subsection{The Reproach}

%   `Oh, you wicked little thing!' cried Alice, catching up the
% kitten, and giving it a little kiss to make it understand that it
% was in disgrace.  `Really, Dinah ought to have taught you better
% manners!  You OUGHT, Dinah, you know you ought!' she added,
% looking reproachfully at the old cat, and speaking in as cross a
% voice as she could manage---and then she scrambled back into the
% arm-chair, taking the kitten and the worsted with her, and began
% winding up the ball again.  But she didn't get on very fast, as
% she was talking all the time, sometimes to the kitten, and
% sometimes to herself.  Kitty sat very demurely on her knee,
% pretending to watch the progress of the winding, and now and then
% putting out one paw and gently touching the ball, as if it would
% be glad to help, if it might.

%   `Do you know what to-morrow is, Kitty?' Alice began.  `You'd
% have guessed if you'd been up in the window with me---only Dinah
% was making you tidy, so you couldn't.  I was watching the boys
% getting in stick for the bonfire---and it wants plenty of
% sticks, Kitty!  Only it got so cold, and it snowed so, they had
% to leave off.  Never mind, Kitty, we'll go and see the bonfire
% to-morrow.'  Here Alice wound two or three turns of the worsted
% round the kitten's neck, just to see how it would look:  this led
% to a scramble, in which the ball rolled down upon the floor, and
% yards and yards of it got unwound again.

%   `Do you know, I was so angry, Kitty,' Alice went on as soon as
% they were comfortably settled again, `when I saw all the mischief
% you had been doing, I was very nearly opening the window, and
% putting you out into the snow!  And you'd have deserved it, you
% little mischievous darling!  What have you got to say for
% yourself?  Now don't interrupt me!' she went on, holding up one
% finger.  `I'm going to tell you all your faults.  Number one:
% you squeaked twice while Dinah was washing your face this
% morning.  Now you can't deny it, Kitty:  I heard you!  What that
% you say?' (pretending that the kitten was speaking.)  `Her paw
% went into your eye?  Well, that's YOUR fault, for keeping your
% eyes open---if you'd shut them tight up, it wouldn't have
% happened.  Now don't make any more excuses, but listen!  Number
% two:  you pulled Snowdrop away by the tail just as I had put down
% the saucer of milk before her!  What, you were thirsty, were you?

\include{chapters/eval}
\ifproject
\chapter{Conclusions and Discussions}

\section{Conclusions}

In conclusion, China's remarkable journey of technology and economic growth 
is a testament to the country's resilience, adaptability, and strategic vision. 
Over the past few decades, China has evolved from a primarily agrarian society 
into a global economic powerhouse with a significant influence on the 
technological landscape. Several key points emerge from this exploration

\subsection{Policy-Driven Transformation}

China's economic and technological growth is largely policy-driven. Government 
initiatives, economic reforms, and investment strategies have played a pivotal 
role in propelling China forward.

\subsection{Global Impact}

China's growth has global implications. Its integration into the global economy, 
as well as its innovations in technology, trade, and infrastructure, affect 
nations and industries around the world.

\subsection{Innovation and Technological Advancements}

China's investments in research and development, intellectual property protection, 
and innovation ecosystems have fueled advancements in sectors like 
telecommunications, artificial intelligence, and renewable energy.

\section{Challenges}

While China's growth offers numerous opportunities, it also presents challenges, such 
as environmental sustainability, income inequality, and geopolitical tensions. 
Managing these challenges is essential for long-term stability.

\section{Suggestions and further improvements}

Here are some suggestions for further improvement and considerations regarding China's 
technology and economic growth:

\subsection{Sustainable Development}

Emphasize sustainable development by prioritizing environmentally friendly technologies 
and practices. This includes reducing air and water pollution, conserving resources, 
and promoting renewable energy sources.

\subsection{Innovation Ecosystem}

Continue to cultivate a vibrant innovation ecosystem by supporting startups, providing 
access to venture capital, and fostering a culture of entrepreneurship.

\subsection{Rural Development}

Address regional disparities by promoting economic development and technological 
advancement in less-developed rural areas, reducing the urban-rural economic divide.
\fi

\bibliography{sampleReport}

\ifproject
\normalspacing
\appendix
\include{chapters/appendix}

%% Display glossary (optional) -- need glossary option.
\ifglossary\glossarypage\fi

%% Display index (optional) -- need idx option.
\ifindex\indexpage\fi

\begin{biosketch}
\begin{center}
  \includegraphics[width=1.5in]{mugshot.jpg}
\end{center}
Your biosketch goes here. Make sure it sits inside
the \texttt{biosketch} environment.
\end{biosketch}
\fi % \ifproject
\end{document}
