\chapter{\ifenglish Background Knowledge and Theory\else ทฤษฎีที่เกี่ยวข้อง\fi}

ในการสร้างเว็บแอพลิเคชันนี้ ทางเราได้มีการศึกษาค้นคว้าทฤษฎีต่างๆที่เกี่ยวข้องกับการสร้างเว็บแอปพลิเคชัน คือ ด้าน Frontend, Backend อีกทั้งได้ศึกษาเกี่ยวกับการทำระบบ Recommendation และ Data Visualization เพื่อให้เว็บของเราน่าใช้งานเพิ่มขึ้น
อีกทั้งยังใช้ความรู้จากวิชา HCI มาช่วยออกแบบตัวเว็บแอปพลิเคชัน


\section{ด้าน Frontend}
\subsection{JavaScript}
๋JavaScript เป็นภาษาโปรแกรมที่ได้รับความนิยมในการใช้พัฒนาเว็บแอปพลิเคชัน โดยความนิยมนี้มาจากหลายสาเหตุ เช่น
การเปลี่ยนแปลงส่วนต่างๆของเว็บโดยที่ไม่ต้องโหลดหน้าใหม่, สามารถใช้งานได้ทั้งฝั่งเว็บบราวเซอร์และเซิร์ฟเวอร์ และมีการแบ่งปันความรู้เกี่ยวกับภาษานี้บนอินเทอร์เน็ตอย่างกว้างขวาง
\subsection{React}
React เป็นไลบรารี JavaScript ที่ใช้ในการพัฒนาเว็บแอปพลิเคชันในฝั่งที่ติดต่อผู้ใช้ (User Interface) โดย Library นี้มีจุดเด่นคือ มีระบบ State ที่ใช้ในการควบคุมสถานะของเว็บได้โดยง่าย
อีกทั้งยังสามารถสร้าง UI components ย่อยๆเพื่อนำมาใช้ซ้ำในหน้าอื่นๆได้ 
\subsection{TypeScript}
Typescript เป็นภาษาโปรแกรมที่ถูกพัฒนาต่อจาก JavaScript โดยเพิ่ม Static typing เพื่อช่วยให้นักพัฒนาสามารถระบุชนิดข้อมูลของตัวแปรและแก้ไขข้อผิดพลาดเกี่ยวกับชนิดของตัวแปรได้ง่าย ซึ่งส่งผลให้การพัฒนา
แอปพลิเคชันขนาดใหญ่และการดูแลหลังการพัฒนาสะดวกมากยิ่งขึ้น

\section{ด้าน Backend}
\subsection{SQL Database}
SQL Database คือโครงสร้างที่ใช้สำหรับการจัดเก็บและจัดการข้อมูลด้วย Structured Query Language (SQL) โดยฐานข้อมูลนี้ประกอบด้วยตาราง(tables) ซึ่งเป็นโครงสร้างข้อมูลที่ถูกกำหนดไว้ล่วงหน้า
แต่ละตารางจะมีคอลัมน์(columns)และแถว(rows)ที่ใช้ในการเก็บข้อมูล ซึ่งคุณสมบัติสำคัญของฐานข้อมูลนี้คือ เราสามารถสร้างความสัมพันธ์ของข้อมูล โดยการเชื่อมโยงข้อมูลจากตารางหนึ่งไปอีกตารางหนึ่งได้
\subsection{Firebase}
Firebase คือแพลตฟอร์มพัฒนาแอปพลิเคชันของ Google ที่ช่วยให้นักพัฒนาสามารถสร้างแอปพลิเคชันขนาดใหญ่ได้ง่ายและรวดเร็วโดยไม่ต้องกังวลเรื่องพื้นฐานการเขียนโปรแกรม
โดย Firebase มี Services ที่น่าสนใจอยู่หลายอย่าง เช่น ระบบยืนยันตัวตน, การจัดเก็บข้อมูลของแอปพลิเคชันบนคลาวน์, การวิเคราะห์ประสิทธิภาพของเว็บแอปพลิเคชัน, การโฮสต์เว็บแอปพลิเคชันบนคลาวน์ เป็นต้น
\subsection{CMU-OAuth}
CMU-OAuth เป็นระบบยืนยันตัวตนนักศึกษาและบุคลากรของมหาวิทยาลัยเชียงใหม่ โดยระบบนี้เป็นส่วนที่ถูกพัฒนามาจาก OAuth (Open Authorization) ซึ่งเป็นโปรโตคอลระบุตัวตนและเป็นตัวกลางที่แอปพลิเคชันใช้ในการเข้าถึงข้อมูลของผู้ใช้ มีจุดเด่นคือ ช่วยให้ผู้ใช้ลงชื่อเข้าใช้หลายบริการ
ได้ด้วยข้อมูลบัญชีเดียว และผู้ใช้ยังสามารถจำกัดสิทธิการเข้าถึงข้อมูลของแต่ละบริการได้เท่าที่จำเป็น
\section{ระบบ Text PreProcessing}
คือ กระบวนการปรับแต่งข้อมูลก่อนที่จะนำข้อมูลนั้นไปวิเคราะห์หรือประมวลผลต่อ สามารถทำได้หลายวิธี เช่น การตัดคำสำคัญแยกออกจากประโยค, ลบตัวอักษรที่ไม่จำเป็น, การหาคำสำคัญ(Keyword) เป็นต้น เพื่อให้สะดวกต่อการนำข้อมูลไปใช้ในการพัฒนาระบบ recommendation

\section{ระบบ Recommendation}
ระบบ Recommendation เป็นระบบคอมพิวเตอร์ที่ออกแบบมาเพื่อแนะนำสิ่งต่างๆให้กับผู้ใช้ โดยใช้วิธีการต่างๆเพื่อแนะนำสิ่งที่เป็นประโยชน์สูงสุด ระบบนี้มักใช้ในแอปพลิเคชันและเว็บไซต์ต่างๆเพื่อแนะนำสินค้า,บริการ,ข่าวสาร และอื่นๆให้กับผู้ใช้
\subsection{Content Based Filtering}
Content Based Filtering เป็นเทคนิคหนึ่งในการแนะนำเนื้อหาให้กับผู้ใช้ วิธีการเบื้องต้นคือ เมื่อผู้ใช้ลงชื่อเข้าใช้งานเว็บแอปพลิเคชันครั้งแรก ทางเว็บจะให้ผู้ใช้เลือก Tag ที่ตัวเองสนใจ แล้วแนะนำสิ่งที่มี Tag เหมือนหรือคล้ายกัน
\subsection{Collaborative Filtering}
Collaborative Filtering เป็นอีกเทคนิคหนึ่งในการแนะนำเนื้อหาให้กับผู้ใช้ วิธีการเบื้องต้นคือ ระบบจะแนะนำสิ่งที่น่าสนใจที่มาจากผู้ใช้อื่นที่มีข้อมูลเหมือนหรือคล้ายกัน เช่น เพศ, อายุ, เงินเดือน เป็นต้น

\section{ความรู้ตามหลักสูตรที่นำมาใช้}
\subsection{ความรู้ด้านHuman Computer Interaction ใช้ในการออกแบบดีไซน์หน้าเว็บให้สื่อประสานกับกลุ่มผู้ใช้งานเป้าหมาย}
\subsection{ความรู้ด้านNatural Language Processing ใช้ประยุกต์ในการพัฒนาระบบแนะนำกิจกรรม}
\subsection{ความรู้ด้านWeb Development ใช้ในการพัฒนาเว็บไซต์}
\subsection{ความรู้ด้านDatabase ใช้ในการออกแบบฐานข้อมูล}
\subsection{ความรู้ด้านInfra and Cloud technology ใช้ในการdeployหน้าเว็บ}

\section{ความรู้นอกหลักสูตรที่นำมาใช้}
\subsection{ความรู้เรื่องการใช้Firebase ในการเก็บข้อมูล}
% \subsection{Subsection heading goes here}

% Subsection 1 text

% \subsubsection{Subsubsection 1 heading goes here}
% Subsubsection 1 text

% \subsubsection{Subsubsection 2 heading goes here}
% Subsubsection 2 text

% \section{Third section}
% Section 3 text. The dielectric constant\index{dielectric constant}
% at the air-metal interface determines
% the resonance shift\index{resonance shift} as absorption or capture occurs
% is shown in Equation~\eqref{eq:dielectric}:

% \begin{equation}\label{eq:dielectric}
% k_1=\frac{\omega}{c({1/\varepsilon_m + 1/\varepsilon_i})^{1/2}}=k_2=\frac{\omega
% \sin(\theta)\varepsilon_\mathit{air}^{1/2}}{c}
% \end{equation}

% \noindent
% where $\omega$ is the frequency of the plasmon, $c$ is the speed of
% light, $\varepsilon_m$ is the dielectric constant of the metal,
% $\varepsilon_i$ is the dielectric constant of neighboring insulator,
% and $\varepsilon_\mathit{air}$ is the dielectric constant of air.

% \section{About using figures in your report}

% % define a command that produces some filler text, the lorem ipsum.
% \newcommand{\loremipsum}{
%   \textit{Lorem ipsum dolor sit amet, consectetur adipisicing elit, sed do
%   eiusmod tempor incididunt ut labore et dolore magna aliqua. Ut enim ad
%   minim veniam, quis nostrud exercitation ullamco laboris nisi ut
%   aliquip ex ea commodo consequat. Duis aute irure dolor in
%   reprehenderit in voluptate velit esse cillum dolore eu fugiat nulla
%   pariatur. Excepteur sint occaecat cupidatat non proident, sunt in
%   culpa qui officia deserunt mollit anim id est laborum.}\par}

% \begin{figure}
%   \centering

%   \fbox{
%      \parbox{.6\textwidth}{\loremipsum}
%   }

%   % To include an image in the figure, say myimage.pdf, you could use
%   % the following code. Look up the documentation for the package
%   % graphicx for more information.
%   % \includegraphics[width=\textwidth]{myimage}

%   \caption[Sample figure]{This figure is a sample containing \gls{lorem ipsum},
%   showing you how you can include figures and glossary in your report.
%   You can specify a shorter caption that will appear in the List of Figures.}
%   \label{fig:sample-figure}
% \end{figure}

% Using \verb.\label. and \verb.\ref. commands allows us to refer to
% figures easily. If we can refer to Figures
% \ref{fig:walrus} and \ref{fig:sample-figure} by name in the {\LaTeX}
% source code, then we will not need to update the code that refers to it
% even if the placement or ordering of the figures changes.

% \loremipsum\loremipsum

% % This code demonstrates how to get a landscape table or figure. It
% % uses the package lscape to turn everything but the page number into
% % landscape orientation. Everything should be included within an
% % \afterpage{ .... } to avoid causing a page break too early.
% \afterpage{
%   \begin{landscape}
%   \begin{table}
%     \caption{Sample landscape table}
%     \label{tab:sample-table}

%     \centering

%     \begin{tabular}{c||c|c}
%         Year & A & B \\
%         \hline\hline
%         1989 & 12 & 23 \\
%         1990 & 4 & 9 \\
%         1991 & 3 & 6 \\
%     \end{tabular}
%   \end{table}
%   \end{landscape}
% }

% \loremipsum\loremipsum\loremipsum

% \section{Overfull hbox}

% When the \verb.semifinal. option is passed to the \verb.cpecmu. document class,
% any line that is longer than the line width, i.e., an overfull hbox, will be
% highlighted with a black solid rule:
% \begin{center}
% \begin{minipage}{2em}
% juxtaposition
% \end{minipage}
% \end{center}

\section{\ifenglish%
\ifcpe CPE \else ISNE \fi knowledge used, applied, or integrated in this project
\else%
ความรู้ตามหลักสูตรซึ่งถูกนำมาใช้หรือบูรณาการในโครงงาน
\fi
}

อธิบายถึงความรู้ และแนวทางการนำความรู้ต่างๆ ที่ได้เรียนตามหลักสูตร ซึ่งถูกนำมาใช้ในโครงงาน

\section{\ifenglish%
Extracurricular knowledge used, applied, or integrated in this project
\else%
ความรู้นอกหลักสูตรซึ่งถูกนำมาใช้หรือบูรณาการในโครงงาน
\fi
}

อธิบายถึงความรู้ต่างๆ ที่เรียนรู้ด้วยตนเอง และแนวทางการนำความรู้เหล่านั้นมาใช้ในโครงงาน
