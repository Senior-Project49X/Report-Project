\chapter{\ifenglish Introduction\else บทนำ\fi}

\section{\ifenglish Paper rationale\else ที่มาของโครงงาน\fi}

Why I choose this topic? Technological change is often seen as something that 
follows its own logic - something we may welcome, or about which we may protest, 
but which we are unable to alter fundamentally. This reader challenges that 
assumption and its distinguished contributors demonstrate that technology is 
affected at a fundamental level by the social context in which it develops. 
General arguments are introduced about the relation of technology to society 
and different types of technology are examined~\cite{mackenzie2011social}.

The book draws on authors from Karl Marx to Cynthia Cockburn to show that 
production technology is shaped by social relations in the workplace. It moves 
on to the technologies of the household and biological reproduction, which are 
topics that male-dominated social science has tended to ignore or trivialise - 
though these are actually of crucial significance where powerful shaping factors 
are at work, normally unnoticed.

Since 1978, China has achieved unprecedented economic growth, but also faces low 
per capita GDP~\cite{GUO2022100982}. To clarify the driving forces behind this situation, we used per 
capita GDP to represent China`s economic growth and performed total factor 
analysis based on 13 variables in 7 socioeconomic dimensions using panel data 
from 30 Chinese provinces over the 40 years since China opened to the west in 
1978. We found similar determinants in different regressions.


\section{\ifenglish Objectives\else วัตถุประสงค์ของโครงงาน\fi}
\begin{enumerate}
    \item Understanding the Historical Context: Explore how China transitioned 
    from a largely agrarian society to a global economic powerhouse~\cite{academicoup}.
    \item Evaluating Economic Growth Metrics: Analyze key economic indicators~\cite{EconGrowthAnalyze}
    such as GDP growth, per capita income and trade balances.
    \item Identifying Key Policy Initiatives: Explore the role of government 
    policies in promoting economic growth and technological innovation in China.
\end{enumerate}

\section{Impacts of this term paper}

Studying the relationship between public R\&D subsidies and private innovation investments in the China 
context offers an interesting case. China`s R\&D investments as a share of GDP nearly doubled from 0.9 
percent to 1.6 percent in just one decade from 2000 to 2010, with more than half of these investments 
coming from large and medium-sized enterprises (LMEs) by 2010~\cite{BOEING2022121212}

\begin{enumerate}
    \item Academic Insights: Research on China`s economic growth provides valuable data and 
    insights for economists studying development, trade, and economic theory.
    \item Technology Policy: Knowledge about China`s technology policies and innovation ecosystem can 
    inform technology and innovation policies in other countries~\cite{surie2017creating}.
    \item Market Entry Strategies: Understanding China's economic growth helps businesses formulate 
    market entry strategies and adapt to evolving consumer preferences and market conditions.
    \item Supply Chain Management: China's role in global supply chains is significant, and businesses 
    need insights into China's economic landscape for effective supply chain management~\cite{li2009china}.
    \item Trade Relations: China's economic growth affects trade relations globally, with implications 
    for multinational corporations, supply chains, and global commerce.
    \item Social Impacts: Understanding China's growth helps researchers assess the social consequences, 
    including income inequality, urbanization, and labor markets~\cite{chen1996regional}.
\end{enumerate}
