\chapter{\ifenglish Historical Context\else ทฤษฎีที่เกี่ยวข้อง\fi}

\section{The Qing Dynasty}

The Qing Dynasty, which ruled China from 1644 to 1912, was a complex period in 
Chinese history, marked by both economic prosperity and challenges. Here is the 
historical context of China's economy during the Qing Dynasty:

\subsection{Manchu Conquest}

The Qing Dynasty was founded by the Manchu people, who originated in 
present-day northeastern China and established their rule after 
defeating the Ming Dynasty in 1644. The early Qing rulers sought to 
consolidate their power and stabilize the empire.

\subsection{Prosperity and Trade}

The Qing Dynasty presided over a period of economic prosperity. Domestic 
trade flourished, as well as foreign trade along the Silk Road, which 
facilitated cultural exchange and the movement of goods and ideas.

\subsection{Agriculture and Rural Economy}

Agriculture remained the backbone of the Chinese economy. The Qing Dynasty 
expanded agricultural land, improved irrigation systems, and promoted new 
crop varieties, which helped to sustain the large population.

In the late 20th and early 21st centuries, rural restructuring has been 
identified in Western Europe, North America, and Israel in the Middle East.
At the same time, such transformational development has also taken place 
in the rural areas of developing countries, such as China, India, the 
Philippines, Zimbabwe, and Ecuador. This rapid and radical rural 
restructuring is often referred to as rural transformation development.
In most developing countries, RTD is usually characterized by changes in 
agricultural intensity, crop selection patterns, farmland, land 
productivity and farm income, labor and technological productivity, 
and major improvements in rural housing and economic and social 
conditions resulting from industrialization and urbanization~\cite{LONG20111094}


\subsection{Population Growth}

During the Qing Dynasty, China experienced significant population growth, 
which put pressure on resources and arable land. The government implemented 
policies to encourage agricultural development.

China is the second most populous country in Asia as well as the second most 
populous country in the world, with a population of 1,425,671,352.
China has an enormous population with a relatively small youth component, 
partially a result of China's one-child policy that was implemented from 
1979 until 2015. As of 2022, Chinese state media reported the country's 
total fertility rate to be 1.09. China was the world's most populous country 
from at least 1950 until being surpassed by India in 2023.

During 1960-2015, the population grew to nearly 1.4 billion. Under Mao Zedong, 
China nearly doubled in population from 540 million in 1949 to 969 million in 
1979. This growth slowed because of the one-child policy instituted in 1979. 
The 2022 data shows a declining population for the first time since 1961.

\subsection{Commercialization and Urbanization}

A growing merchant class and increasing urbanization characterized the era. 
Prosperous trade and commerce led to the growth of cities and the emergence 
of a consumer culture.

Since then China has been transformed. From the 1970s up until the end of 
2011, when half of China's population lived in cities, around 500 million 
people have added to China's urban population in the last three years—and 
the process continues to happen at a terrific rate.

China's urbanization drive to date has wrought severe social and environmental 
problems though, and a new approach is needed. Industries that are key to the 
process, such as construction, have thus far tooled themselves to deliver 
quantity over quality, raising the question: Can China actually build the 
'new China'?

\subsection{Opium Trade and Foreign Influence}

The agreements reached between the Western powers and China following the 
Opium Wars came to be known as the “unequal treaties” because in practice 
they gave foreigners privileged status and extracted concessions from the 
Chinese. Ironically, the Qing Government had fully supported the clauses 
on extraterritoriality and most-favored nation status in the first treaties 
in order to keep the foreigners in line. This treaty system also marked a 
new direction for Chinese contact with the outside world. For years, the 
Chinese had conducted their foreign policy through the tribute system, in 
which foreign powers wishing to trade with China were required first to 
bring a tribute to the emperor, acknowledging the superiority of Chinese 
culture and the ultimate authority of the Chinese ruler. Unlike China's 
neighbors, the European powers ultimately refused to make these 
acknowledgements in order to trade, and they demanded instead that China 
adhere to Western diplomatic practices, such as the creation of treaties. 
Although the unequal treaties and the use of the most-favored-nation 
clause were effective in creating and maintaining open trade with China, 
both were also important factors in building animosity and resentment 
toward Western imperialism.

\subsection{Foreign Concessions}

As a result of military defeats and unequal treaties, foreign powers gained 
control over areas in China known as concessions, where they enjoyed 
extraterritorial rights and economic privileges.

Concessions in China were a group of concessions that existed during the 
late Imperial China and the Republic of China, which were governed and 
occupied by foreign powers, and are frequently associated with colonialism 
and imperialism.

The concessions had extraterritoriality and were enclaves inside key cities 
that became treaty ports. All the concessions have been dissolved in the 
present day.

\subsubsection{Imperial China period}

Imperial China granted the concessions during the latter period of the Qing 
dynasty, as a result of the series of "unequal treaties". They began in 1842's 
Treaty of Nanjing with the United Kingdom. Under each treaty, China was 
usually obligated to open more treaty ports for trade and lease out more 
territory as part of the concession or surrender it completely. The one 
exception that preceded this period was Macau, which had been leased in 
1557 to the Kingdom of Portugal, during the Ming dynasty; Portugal continued 
to pay rent to China up to 1863 to stay in Macau.

\subsubsection{Republic of China period}

The foreign concessions continued to exist during the mainland period of the 
Republic of China. The Asia and Pacific theatre of the First World War would 
be another major incident changing the ownership of concessions in China with 
Japanese expansion. Concessions were partially curtailed in the Washington 
Naval Treaty and the Nine Power Treaty attempting to reaffirm the 
sovereignty of China.

Many foreigners arrived in the cities aiming primarily to get rich. During 
the first phase of the Chinese Civil War in the 1920s, the concessions saw 
a sharp increase in immigration both from surrounding Chinese territory, 
and from the West and Japan. The population of Chinese residents eventually 
surpassed foreigners inside the concessions. With international travelers, 
culture took on an eclectic character of many influences—including both 
language and architecture. This effect was exemplified in the Shanghai 
International Settlement and the multi-concessions in Tianjin. Writings from 
the time period indicate that both the Prussians and Russians were seen as 
acting culturally British. The wealthy built opulent buildings with multiple 
European and Chinese inspirations. Some Chinese entrepreneurs became very 
wealthy and hired foreign designers and architects.

\section{Modern China}

In the first years following the founding of the People`s Republic of China, 
a S\&T (Science and technology) system was built that was characterized by 
central governance and regulation by unitary plan~\cite{historicalSandT}.

\subsection{The first stage (1985-1992): stabling one part and setting free the vast}

During this stage, the guideline of scientific and technological development
was "facing" and "replying on," that is to say, S\&T facing economic construction, 
economic construction relying on S\&T and the main policy is "activating the 
research institutions and relevant personnel".

\subsection{The second stage (1992-1998): stabilizing one side and setting free the vast}

In 1992, Mr Deng Xiaoping launched a new stage for China's economic system, an 
era of socialist market economy. At the stage, the direction of the S\&T system 
reform was adjusted to "facing," "relying on," and "scaling new heights"; the 
main policy direction was based on "Stabilizing one side and setting free a 
vastness," to assign to personnel, adjust the structure and encourage economic 
involvement in the development of S\&T.

\subsection{The third stage (1998-2006): building a national innovation system}

Throughout this stage, S\&T development strategies and reform were substantially 
adjusted, and “the strategy of prospering the nation with science and education” 
became a national strategy. As a matter of fact, the strategy has become the 
main task of the government. Strengthening the national innovation system and 
speeding up the industrialization of scientific and technological achievements 
therefore became the main policy direction during this period.

\subsection{The fourth stage (2006-now): enhancing the capacity of independent innovation and building an innovation-oriented country}

President Hu Jintao`s speech at the National Science and Technology Conference 
marked S\&T and innovation starting to become key drivers of the country 
development model, with the industrial structure being adjusted, and economic 
and social growth. Policies` focus became to target building the innovation 
capacity, implementing the national mid and long-term technology development 
plan, accelerating production and research co-operation, promoting the S\&T 
transfer and nurturing new industries.