\documentclass[semifinal,isne]{cpecmu}

%% This is a sample document demonstrating how to use the CPECMU
%% project template. If you are having trouble, see "cpecmu.pdf" for
%% documentation.

\projectNo{1}
\acadyear{2023}

\titleTH{เทคโนโลยีของจีนและการเติบโตทางเศรษฐกิจของจีน}
\titleEN{China`s Technology and Economic Growth of China}

\author{นายเนวิน ยามากุชิ}{Newin Yamaguchi}{630615028}

\cpeadvisor{nisit}

%% Some possible packages to include:
\usepackage[final]{graphicx} % for including graphics

%% Add bookmarks and hyperlinks in the document.
\PassOptionsToPackage{hyphens}{url}
\usepackage[colorlinks=true,allcolors=Blue4,citecolor=red,linktoc=all]{hyperref}
\def\UrlLeft#1\UrlRight{$#1$}

%% Needed just by this example, but maybe not by most reports
\usepackage{afterpage} % for outputting
\usepackage{pdflscape} % for landscape figures and tables. 

%% Some other useful packages. Look these up to find out how to use
%% them.
% \usepackage{natbib}    % for author-year citation styles
% \usepackage{txfonts}
% \usepackage{appendix}  % for appendices on a per-chapter basis
% \usepackage{xtab}      % for tables that go over multiple pages
% \usepackage{subfigure} % for subfigures within a figure
% \usepackage{pstricks,pdftricks} % for access to special PostScript and PDF commands
% \usepackage{nomencl}   % if you have a list of abbreviations

%% if you're having problems with overfull boxes, you may need to increase
%% the tolerance to 9999
% \tolerance=9999

\bibliographystyle{plain}
% \bibliographystyle{IEEEbib}

% \renewcommand{\topfraction}{0.85}
% \renewcommand{\textfraction}{0.1}
% \renewcommand{\floatpagefraction}{0.75}

%% Example for glossary entry
%% Need to use glossary option
%% See glossaries package for complete documentation.
\ifglossary
  \newglossaryentry{lorem ipsum}{
    name=lorem ipsum,
    description={derived from Latin dolorem ipsum, translated as ``pain itself''}
  }
\fi

%% Uncomment this command to preview only specified LaTeX file(s)
%% imported with \include command below.
%% Any other file imported via \include but not specified here will not
%% be previewed.
%% Useful if your report is large, as you might not want to build
%% the entire file when editing a certain part of your report.
% \includeonly{chapters/intro,chapters/background}

\begin{document}
\maketitle
% \makesignature

\ifproject
\begin{abstractTH}
เขียนบทคัดย่อของโครงงานที่นี่

การเขียนรายงานเป็นส่วนหนึ่งของการทำโครงงานวิศวกรรมคอมพิวเตอร์
เพื่อทบทวนทฤษฎีที่เกี่ยวข้อง อธิบายขั้นตอนวิธีแก้ปัญหาเชิงวิศวกรรม และวิเคราะห์และสรุปผลการทดลองอุปกรณ์และระบบต่างๆ
\enskip อย่างไรก็ดี การสร้างรูปเล่มรายงานให้ถูกรูปแบบนั้นเป็นขั้นตอนที่ยุ่งยาก
แม้ว่าจะมีต้นแบบสำหรับใช้ในโปรแกรม Microsoft Word แล้วก็ตาม
แต่นักศึกษาส่วนใหญ่ยังคงค้นพบว่าการใช้งานมีความซับซ้อน และเกิดความผิดพลาดในการจัดรูปแบบ กำหนดเลขหัวข้อ และสร้างสารบัญอยู่
\enskip ภาควิชาวิศวกรรมคอมพิวเตอร์จึงได้จัดทำต้นแบบรูปเล่มรายงานโดยใช้ระบบจัดเตรียมเอกสาร
\LaTeX{} เพื่อช่วยให้นักศึกษาเขียนรายงานได้อย่างสะดวกและรวดเร็วมากยิ่งขึ้น
\end{abstractTH}

\begin{abstract}
The abstract would be placed here. It usually does not exceed 350 words
long (not counting the heading), and must not take up more than one (1) page
(even if fewer than 350 words long).

Make sure your abstract sits inside the \texttt{abstract} environment.
\end{abstract}

\iffalse
\begin{dedication}
This document is dedicated to all Chiang Mai University students.

Dedication page is optional.
\end{dedication}
\fi % \iffalse

\begin{acknowledgments}
Your acknowledgments go here. Make sure it sits inside the
\texttt{acknowledgment} environment.

\acksign{2020}{5}{25}
\end{acknowledgments}%
\fi % \ifproject

\contentspage

\ifproject
\figurelistpage

\tablelistpage
\fi % \ifproject

% \abbrlist % this page is optional

% \symlist % this page is optional

% \preface % this section is optional


\pagestyle{empty}\cleardoublepage
\normalspacing \setcounter{page}{1} \pagenumbering{arabic} \pagestyle{cpecmu}

\chapter{\ifenglish Introduction\else บทนำ\fi}

\section{\ifenglish Project rationale\else ที่มาของโครงงาน\fi}
มีบุคลากรภายในมหาวิทยาลัยเชียงใหม่หลายคนต้องการที่จะหาคนมาเข้าร่วมงานอีเว้นท์ หรือ เข้ามาช่วยในงานโครงงานโปรเจคต่างๆ เเต่ไม่สามารถหาผู้เข้าร่วมได้ ซึ่งในหลายๆครั้งนั้นอาจมีนักศึกษาหรือบุคลากรจำนวนมากที่สนใจเข้าร่วมเเต่ไม่ได้เข้าร่วม เพียงเพราะไม่ได้รับข่าวสารการประกาศ ซึ่งอาจเป็นเพราะด้วยช่องทางที่ผู้จัดประกาศนั้นเข้าไม่ถึงบุคลากรเหล่านั้นด้วยเหตุผลต่างๆอาทิเช่น ประกาศในโซเชียลมีเดียเเล้วผู้ที่สนใจไม่เห็นเนื่องด้วยอาจไม่ได้ติดตามช่องทางที่ประกาศหรืออาจเพราะถูกบดบังด้วยอัลกอริทึมของโซเชียลมีเดียนั้นๆ


\section{\ifenglish Objectives\else วัตถุประสงค์ของโครงงาน\fi}
\begin{enumerate}
    \item พัฒนาเว็บที่สามารถรองรับการประกาศกิจกรรม เเละหัวข้อ senior projectต่างๆ ให้ผู้ที่สนใจนั้นสามารถเข้ามามีส่วนร่วมกันได้อย่างมีประสิทธิภาพ
    \item พัฒนาเว็บที่นำความรู้ทางด้านData Analytic มาใช้ในการเเนะนำกิจกรรม หรือหัวข้อ senior project ให้เเก่ผู้ใช้โดยกิจกรรมที่เเนะนำจะต้องสอดคล้องกับความสนใจของผู้ใช้รายนั้นๆ
\end{enumerate}

\section{\ifenglish Project scope\else ขอบเขตของโครงงาน\fi}

\subsection{\ifenglish Hardware scope\else ขอบเขตด้านฮาร์ดแวร์\fi}
\begin{enumerate}
    \item คอมพิวเตอร์เพื่อใช้พัฒนาเว็บแอปพลิเคชันและตรวจสอบผลลัพธ์ผ่านเว็บบราวเซอร์
    \item สมาร์ทโฟนระบบแอนดรอยด์เพื่อใช้ตรวจสอบผลลัพธ์ผ่านเว็บบราวเซอร์
\end{enumerate}
\subsection{\ifenglish Software scope\else ขอบเขตด้านซอฟต์แวร์\fi}
\begin{enumerate}
    \item การเข้าถึงเว็บแอพลิเคชัน สามารถเข้าผ่านเว็บบราวเซอร์ต่างๆ เช่น Chrome, Firefox เป็นต้น
    \item ส่วนบัญชีผู้ใช้ คือการยืนยันตัวตนผ่าน CMU-OAuth
    \item ส่วนแสดงกิจกรรมทั้งหมด ผู้ใช้จะสามารถดูข้อมูลเบื้องต้นของกิจกรรมต่างๆได้ โดยในส่วนนี้จะแบ่งเป็นกิจกรรมที่มีผู้สนใจเยอะ และกิจกรรมทั้งหมด
    \item ส่วนการสร้างกิจกรรม ผู้ใช้สามารถสร้างกิจกรรมใหม่ขึ้นมา โดยระบุรายละเอียดต่างๆของกิจกรรม สามารถเลือกได้ว่าจะต้องการผู้สมัครเข้าร่วมกิจกรรมหรือไม่
    \item ส่วนแสดงกิจกรรมเฉพาะ เมื่อผู้ใช้เข้ามาส่วนนี้ ผู้ใช้จะสามารถดูข้อมูลของกิจกรรมได้โดยละเอียด และสามารถสมัครเข้าร่วมกิจกรรมได้
    \item ส่วนตอบรับการเข้าร่วมกิจกรรม ผู้ที่สร้างกิจกรรมสามารถเลือกได้ว่าจะให้ผู้สมัครคนไหนมีสิทธิเข้าร่วมกิจกรรมบ้าง
    \item ส่วนการให้คะแนนกิจกรรม ผู้เข้าร่วมกิจกรรมสามารถให้คะแนนกิจกรรมและผู้จัดได้ ส่วนผู้จัดก็สามารถให้คะแนนผู้เข้าร่วมได้เช่นกัน
    \item ส่วนแดชบอร์ด ผู้ใช้สามารถดูสถิติต่างๆที่ตนเองสนใจได้ เช่น ผู้สร้างกิจกรรมสามารถดูผลตอบรับของผู้ใช้คนอื่นๆ, ผู้ดูแลระบบสามารถดูสถิติโดยรวมของเว็บแอพลิเคชันได้ เป็นต้น
\end{enumerate}

\subsection{ขอบเขตด้านกลุ่มผู้ใช้}
นักศึกษาและบุคลากรของมหาวิทยาลัยเชียงใหม่ที่มี CMU-Account

\subsection{ขอบเขตด้านข้อมูล}
\begin{enumerate}
    \item กิจกรรมประเภทต่างๆ เช่น รับน้องขึ้นดอย, CPE Music box, จับกลุ่มออกกำลังกาย เป็นต้น
    \item ข้อมูลของผู้ใช้ที่ได้รับจาก CMU-Account
\end{enumerate}

\section{\ifenglish Expected outcomes\else ประโยชน์ที่ได้รับ\fi}
\begin{enumerate}
    \item สามารถทำให้กิจกรรมต่างๆที่มาฝากประกาศในช่องทางเรา เข้าถึงกลุ่มเป้าหมายได้มากขึ้น
    \item สามารถทำให้นักศึกษาและบุคลากรในมหาวิทยาลัย ได้เห็นกิจกรรมที่ตัวเองสนใจได้ง่ายขึ้น
    \item สามารถทำให้การหาข้อมูลกิจกรรมต่างๆนั้น สะดวกมากยิ่งขึ้น
\end{enumerate}

\section{\ifenglish Technology and tools\else เทคโนโลยีและเครื่องมือที่ใช้\fi}

\subsection{\ifenglish Hardware technology\else เทคโนโลยีด้านฮาร์ดแวร์\fi}
\begin{enumerate}
    \item ASUS Vivobook Pro 15 : สำหรับพัฒนาเว็บแอพลิเคชัน
    \item Huawei P20 Pro : สำหรับตรวจสอบการแสดงผลบนสมาร์ทโฟน
\end{enumerate}
\subsection{\ifenglish Software technology\else เทคโนโลยีด้านซอฟต์แวร์\fi}
\begin{enumerate}
    \item Figma : เว็บแอพลิเคชันที่ใช้ในการออกแบบ Prototype ของเว็บไซต์
    \item Jira Software : เว็บแอพลิเคชันที่ใช้ในการวางแผนงาน, แบ่งงาน และดูความคืบหน้าของแต่ละงาน
    \item GitHub : Version control ที่สามารถเก็บไฟล์ได้บนอินเทอร์เน็ต
    \item Visual Studio Code : Text Editor ที่ใช้ในการเขียนโค้ด โดยมีจุดเด่นคือมีส่วนขยายโปรแกรมที่สร้างโดยผู้ใช้ทั่วโลก
    \item React : Javascript Library ที่ช่วยในการสร้าง User interface
    \item TypeScript : ภาษาโปรแกรมที่พัฒนาต่อมาจาก Javascript โดยเพิ่ม Static typing เพื่อตรวจสอบความผิดพลาดของโปรแกรมได้โดยง่าย
    \item MySQL : ฐานข้อมูล
    \item Firebase : แพลตฟอร์มที่ใช้พัฒนา backend และจัดการฐานข้อมูลผ่านเว็บบราวเซอร์
\end{enumerate}
\section{\ifenglish Project plan\else แผนการดำเนินงาน\fi}

\begin{plan}{6}{2023}{3}{2024}
    \planitem{6}{2023}{6}{2023}{ค้นหาหัวข้อที่สนใจและอาจารย์ที่ปรึกษา}
    \planitem{6}{2023}{8}{2023}{ค้นหาข้อมูล ทฤษฎีที่เกี่ยวข้องและกำหนดขอบเขต}
    \planitem{7}{2023}{8}{2023}{ออกแบบ Mockup คร่าวๆของเว็บด้วย Figma}
    \planitem{8}{2023}{8}{2023}{ออกแบบ Diagram ของระบบแบบคร่าวๆ}
    \planitem{8}{2023}{9}{2023}{หาข้อมูลเกี่ยวกับกิจกรรมตัวอย่าง}
    \planitem{8}{2023}{10}{2023}{ออกแบบ Flow ของระบบ}
    \planitem{8}{2023}{10}{2023}{ออกแบบ UX/UI ของเว็บด้วย Figma}
    \planitem{9}{2023}{10}{2023}{เขียนรายงานและนำเสนอ 261491}
    \planitem{10}{2023}{10}{2023}{ศึกษา Algorithm สำหรับระบบ Recommendation}
    \planitem{10}{2023}{10}{2023}{ศึกษาการทำ Data Visualization สำหรับหน้าแดชบอร์ด}
    \planitem{10}{2023}{11}{2023}{ออกแบบฐานข้อมูล}
    \planitem{11}{2023}{2}{2024}{พัฒนาเว็บแอพลิเคชัน}
    \planitem{11}{2023}{2}{2024}{ทดสอบกับผู้ใช้จริงและปรับปรุงระบบ}
    \planitem{1}{2024}{3}{2024}{เขียนรายงานและนำเสนอ 261492}

\end{plan}

\section{\ifenglish Roles and responsibilities\else บทบาทและความรับผิดชอบ\fi}
\begin{enumerate}
    \item ส่วนที่ทำงานร่วมกันได้แก่ การวางแผนงาน, การค้นหาความรู้และทฤษฎีที่เกี่ยวข้อง และการพัฒนาเว็บแอพลิเคชัน
    \item ส่วนที่รับผิดชอบโดยนาย ณัฏฐพล ตันจอ 620610786 ได้แก่ การออกแบบหน้าสร้างกิจกรรมและหน้าเข้าร่วมกิจกรรม, รายงานหัวข้อ(1.3, 1.5, 1.6, 2.1-2.3, 3.1-3.3) 
    \item ส่วนที่รับผิดชอบโดยนาย นายธิษณ์ธนัย แก้วเพ็ชร์ 630610741 ได้แก่ การออกแบบหน้าแรกและหน้าแสดงกิจกรรม, รายงานหัวข้อ(บทนำ, 1.1, 1.2, 1.4, 1.8, 3.4-3.5)
\end{enumerate}

\section{\ifenglish%
Impacts of this project on society, health, safety, legal, and cultural issues
\else%
ผลกระทบด้านสังคม สุขภาพ ความปลอดภัย กฎหมาย และวัฒนธรรม
\fi}

แนวทางและโยชน์ในการประยุกต์ใช้งานโครงงานกับงานในด้านอื่นๆ รวมถึงผลกระทบในด้านสังคมและสิ่งแวดล้อมจากการใช้ความรู้ทางวิศวกรรมที่ได้

\chapter{\ifenglish Historical Context\else ทฤษฎีที่เกี่ยวข้อง\fi}

The historical context of China's technology is a rich and multifaceted story 
that spans thousands of years. We can separate historical periods and 
developments that provide insight information.

\section{Late 20th Century to Present}

\begin{itemize}
  \item China's rapid economic growth has been accompanied by significant 
  advancements in technology across various sectors, including 
  telecommunications, manufacturing, and space exploration.
  \item China has become a global leader in areas such as 5G technology, 
  e-commerce, and renewable energy.
\end{itemize}

\section{21st Century Challenges and Ambitions}

\begin{itemize}
  \item China has set ambitious goals for technological innovation, including 
  initiatives like "Made in China 2025" aimed at becoming a global leader in 
  advanced manufacturing and technology.
  \item The Belt and Road Initiative seeks to promote infrastructure 
  development and technology transfer on a global scale.
\end{itemize}

\subsection{Subsection heading goes here}

Subsection 1 text

\subsubsection{Subsubsection 1 heading goes here}
Subsubsection 1 text

\subsubsection{Subsubsection 2 heading goes here}
Subsubsection 2 text

\section{Third section}
Section 3 text. The dielectric constant\index{dielectric constant}
at the air-metal interface determines
the resonance shift\index{resonance shift} as absorption or capture occurs
is shown in Equation~\eqref{eq:dielectric}:

\begin{equation}\label{eq:dielectric}
k_1=\frac{\omega}{c({1/\varepsilon_m + 1/\varepsilon_i})^{1/2}}=k_2=\frac{\omega
\sin(\theta)\varepsilon_\mathit{air}^{1/2}}{c}
\end{equation}

\noindent
where $\omega$ is the frequency of the plasmon, $c$ is the speed of
light, $\varepsilon_m$ is the dielectric constant of the metal,
$\varepsilon_i$ is the dielectric constant of neighboring insulator,
and $\varepsilon_\mathit{air}$ is the dielectric constant of air.

\section{About using figures in your report}

% define a command that produces some filler text, the lorem ipsum.
\newcommand{\loremipsum}{
  \textit{Lorem ipsum dolor sit amet, consectetur adipisicing elit, sed do
  eiusmod tempor incididunt ut labore et dolore magna aliqua. Ut enim ad
  minim veniam, quis nostrud exercitation ullamco laboris nisi ut
  aliquip ex ea commodo consequat. Duis aute irure dolor in
  reprehenderit in voluptate velit esse cillum dolore eu fugiat nulla
  pariatur. Excepteur sint occaecat cupidatat non proident, sunt in
  culpa qui officia deserunt mollit anim id est laborum.}\par}

\begin{figure}
  \centering

  \fbox{
     \parbox{.6\textwidth}{\loremipsum}
  }

  % To include an image in the figure, say myimage.pdf, you could use
  % the following code. Look up the documentation for the package
  % graphicx for more information.
  % \includegraphics[width=\textwidth]{myimage}

  \caption[Sample figure]{This figure is a sample containing \gls{lorem ipsum},
  showing you how you can include figures and glossary in your report.
  You can specify a shorter caption that will appear in the List of Figures.}
  \label{fig:sample-figure}
\end{figure}

Using \verb.\label. and \verb.\ref. commands allows us to refer to
figures easily. If we can refer to Figures
\ref{fig:walrus} and \ref{fig:sample-figure} by name in the {\LaTeX}
source code, then we will not need to update the code that refers to it
even if the placement or ordering of the figures changes.

\loremipsum\loremipsum

% This code demonstrates how to get a landscape table or figure. It
% uses the package lscape to turn everything but the page number into
% landscape orientation. Everything should be included within an
% \afterpage{ .... } to avoid causing a page break too early.
\afterpage{
  \begin{landscape}
  \begin{table}
    \caption{Sample landscape table}
    \label{tab:sample-table}

    \centering

    \begin{tabular}{c||c|c}
        Year & A & B \\
        \hline\hline
        1989 & 12 & 23 \\
        1990 & 4 & 9 \\
        1991 & 3 & 6 \\
    \end{tabular}
  \end{table}
  \end{landscape}
}

\loremipsum\loremipsum\loremipsum

\section{Overfull hbox}

When the \verb.semifinal. option is passed to the \verb.cpecmu. document class,
any line that is longer than the line width, i.e., an overfull hbox, will be
highlighted with a black solid rule:
\begin{center}
\begin{minipage}{2em}
juxtaposition
\end{minipage}
\end{center}

\section{\ifenglish%
\ifcpe CPE \else ISNE \fi knowledge used, applied, or integrated in this project
\else%
ความรู้ตามหลักสูตรซึ่งถูกนำมาใช้หรือบูรณาการในโครงงาน
\fi
}

อธิบายถึงความรู้ และแนวทางการนำความรู้ต่างๆ ที่ได้เรียนตามหลักสูตร ซึ่งถูกนำมาใช้ในโครงงาน

\section{\ifenglish%
Extracurricular knowledge used, applied, or integrated in this project
\else%
ความรู้นอกหลักสูตรซึ่งถูกนำมาใช้หรือบูรณาการในโครงงาน
\fi
}

อธิบายถึงความรู้ต่างๆ ที่เรียนรู้ด้วยตนเอง และแนวทางการนำความรู้เหล่านั้นมาใช้ในโครงงาน

\chapter{Literature Review}

Many studies show China`technology and economic growth of 
china are rapidly developing. Moreover, China`s rapid economic 
and technological growth can be attributed to a combination of 
various factors, policies, and strategic decisions. 

\section{The impact of China`s R\&D subsidies on R\&D investment, technological upgrading and economic growth}

A key argument in favor of state-intervention relates to the idea that 
investments in innovation are limited by financial constraints facing 
firms,3 especially ones from transitioning economy countries~\cite{BOEING2022121212}.

To better harness the growth-enhancing power of innovation, an 
important question that naturally arises for policy-makers is 
how deeply should the state intervene in promoting a country's 
own technological capabilities. Stemming from the failed import 
substitution policies of the 1970s, conventional wisdom calls for 
a rather limited role of the government to support indigenous 
innovation efforts given the public good nature of research and 
development. 2Yet, the explosion of innovation activities in 
emerging economies coinciding with periods of rapid economic 
growth has led to a renewed optimism that state-led innovation 
can be a major contribution to stimulate regional innovation 
systems and national competitive advantage.

\section{China`s changing political landscape: prospects for democracy}

Although each chapter makes equally important and meaningful 
contributions to this impressively coherent volume, readers 
will notice that agency, in the process of democratization, 
is the most salient issue.

Thirty years ago Deng Xiaoping launched his policy of 
“Reform and Opening.” In time, his decision would transform China 
economically, socially, legally, ideologically, and politically, 
no less than Mao's revolution did in 1949. The changes unleashed 
by Deng are difficult to overstate; they did nothing less than 
bring China for the first time fully into the modern world. The 
result is the nation of today's headlines: the third largest 
economy in the world; a land of 200 million Internet users and 
500 million cell phones; a significant actor in some of the most 
pressing international concerns (North Korea, Iran, Africa).

\section{Regional Income Inequality and Economic Growth in China}

Convergence is conditional on physical investment share, employment 
growth, human-capital investment, foreign direct investment, and 
coastal location. We project that, in the near term, overall regional 
inequality as measured by the coefficient of variation is likely to 
decline modestly but that the coast/noncoast income differential is 
likely to increase somewhat~\cite{chen1996regional}

China's spatial income inequality can be defined by inequality 
among regions and urban-rural income disparity. Certain regions, 
especially in eastern China, have more disproportionate advantages 
from the reform and opening up because of preferential policies, 
natural endowment, and improved infrastructure. Compared with the 
central and western areas, the income level in the east is higher, 
resulting in income inequality among regions. Inequality among 
regions and urban-rural income disparity are not entirely 
independent, suggesting that the difference in regional development 
also promotes the further differentiation of urban and rural 
development levels.
\chapter{\ifproject%
\ifenglish Basic Conceptual framework\else การทดลองและผลลัพธ์\fi
\else%
\ifenglish System Evaluation\else การประเมินระบบ\fi
\fi}

ในบทนี้จะทดสอบเกี่ยวกับการทำงานในฟังก์ชันหลักๆ

\chapter{Key Supporting Arguments}

Here are some key supporting arguments regarding China's 
technology and economic growth

\section{Export-Led Growth}

China's economic growth has been significantly driven by its 
exports. The country has become known as the "world's factory," 
manufacturing a wide range of goods for global markets. This 
export-oriented strategy has boosted economic development 
and job creation.

\section{Investment in Infrastructure}

China has invested heavily in infrastructure development. 
Projects like high-speed rail networks, airports, and ports 
have improved connectivity, making it easier to transport 
goods and people, thereby enhancing economic efficiency.

China has accelerated infrastructure investment in the first 
quarter of this year to propel economic growth, launching 
more than 10,000 projects throughout the country.
Analysts estimated that infrastructure investment grew 10 
percent year-on-year in the first three months, driving up 
activity of many associated downstream enterprises and broad 
market demand for basic materials.
According to incomplete statistics, 14 provinces had announced 
data on major projects for the first quarter as of Monday, 
launching a total of 12,571 major projects in sectors including 
transport, water conservation, advanced manufacturing, modern 
services and new types of infrastructure.
The combined investment reached approximately 7 trillion yuan 
(\$1.03 trillion), according to media reports.


\section{Innovation and Research \& Development}

China has increased investments in research and development 
(R\&D), leading to innovations in areas like telecommunications, 
artificial intelligence, and renewable energy. This commitment 
to innovation has enabled China to compete on the global stage 
in technology and other high-value sectors.

China has leaned on its manufacturing prowess for decades to 
support economic development, but it is increasingly seeking 
to contend with countries whose economies are deeply rooted in 
innovation-based growth. China has made considerable progress 
in establishing itself as a pioneer in emerging industries and 
its leaders are increasingly looking toward innovation as a 
driver of its economic growth.

\section{Government Policies}

The Chinese government has played a significant role in fostering 
economic growth and technological advancement. Policies such as 
"Made in China 2025" and "Belt and Road Initiative" have been 
instrumental in guiding China's development and influence on the 
world stage.

In the city of Shanghai, a few churches conduct daily services 
for the faithful, just as churches all over the world do. However, 
China's Patriotic Catholic Association doesn't operate under the 
auspices of the Roman Catholic Church, which the Chinese government 
has banned. It is controlled by a state agency, the Religious 
Affairs Bureau. That's how the Chinese government deals with 
foreign organizations, be they churches or companies. They are 
tolerated in China but can operate only under the state's 
supervision. They can bring in their ideas if they deliver 
value to the country, but their operations will be circumscribed 
by China's goals. If the value—or danger—from them is high, the 
government will create hybrid organizations that it can better 
control. This approach, which never ceases to shock foreigners, 
guides those who are boldly fashioning a new China.

\section{Global Trade and Integration}

China's active participation in global trade, membership in 
international organizations like the World Trade Organization, 
and its position in global supply chains have enhanced its economic 
growth and global influence.

China's engagement in the so-called international fragmentation of 
production - namely 'cross-border dispersion of component 
production/assembly within vertically integrated manufacturing 
industries' - has become an increasingly important form of its 
economic integration into the regional as well as the global 
economy. The paper presents the recent trend of trade in parts 
and components between China and its main trading partners. 
Applying an adjusted gravity modelling method, the paper explores 
how China's pattern of trade in parts and components is being 
determined. The paper found that China's rapid economic growth, 
increasing market size and economies of scale, foreign direct 
investment and infrastructure development including transportation 
and telecommunications are important factors in explaining China's 
rapid increase of bilateral trade in parts and components with its 
trading partners. The paper also found that the spatial distance 
and transportation costs have significant negative impacts on China's 
trade of parts and components suggesting that the reduction in 
transportation costs by technological innovation and investment 
could enhance trade in parts and components, and thereby deepen 
the process of international specialization involving China and 
its main trading partners. The paper argues that given the 
prospects of the rapid growth of the Chinese economy, its current 
and planned massive investments in R\&D and in infrastructure, its 
continual policies in attracting FDI and its rapid move towards 
liberalizing its services sectors including its financial sectors, 
the scope for China and its trading partners to benefit from the 
process of international fragmentation of production is tremendous.

\section{Education and Workforce}

China has invested in education and skills development, resulting 
in a highly skilled and competitive workforce. This has attracted 
foreign companies and stimulated domestic innovation.

After decades of reform, China today has an education system that 
serves the industrial economy well although gaps in access, quality, 
and relevance in education still need to be plugged. However, there 
is now an even larger challenge to meet: delivering the skills needed 
for a modern, digital, postindustrial economy, while instilling a 
new national ethos of lifelong learning, and ensuring that the system 
is equitable. Nothing less than a transformation of China's education 
and skills-development system appears necessary. China has undertaken 
transformative reform before; it now needs to do so again.

\section{Urbanization}

China has experienced massive urbanization, leading to the growth 
of megacities and urban clusters. Urbanization has driven economic 
activity and increased consumer demand, contributing to 
overall growth.

By the same token, urbanization rarely exceeded ten percent of the 
total population although large urban centres were established. For 
example, during the Song, the northern capital Kaifeng 
(of the Northern Song) and southern capital Hangzhou 
(of the Southern Song) had 1.4 million and one million inhabitants, 
respectively. In addition, it was common that urban residents 
also had one foot in the rural sector due to private landholding 
property rights.

In 1949, the year that the People's Republic of China was founded, 
less than 10\% of the population in mainland China was urban. 
Few cities at that time could be considered modern.

\section{Technological Dominance}

China has gained global dominance in specific technological areas, 
such as 5G technology and electric vehicles, further propelling its 
economic growth and influence.

Building technological innovation is a gradual and cumulative process 
driven by industrial R\&D. China has a relatively short history of 
industrial innovation, which is path-dependent. For this reason, 
China has few advantages in established industries such as 
semiconductors and pharmaceuticals, where Western incumbents hold 
'patent thickets' that curb China's catch-up. While China contributed 
27.5 per cent to total global R\&D expenditures in 2022 against the 
United States' 35.6 per cent, US technology giants still dominate 
research and innovation in critical technologies such as artificial 
intelligence.
\chapter{Policy or Recommendation}

\section{Innovation and Intellectual Property Protection}

Strengthen intellectual property protection laws and enforcement 
to encourage innovation and protect the rights of innovators.

\section{Education and Workforce Development}

Invest in education and workforce development to ensure a steady 
supply of skilled labor to support technological advancements 
and economic growth.

\section{Foreign Investment Incentives}

Continue to offer incentives for foreign investment in research 
and development, technology transfer, and joint ventures, 
fostering collaboration and technology exchange.

\section{Technology Transfer and Collaboration}

Encourage partnerships between domestic and foreign companies, 
research institutions, and universities to facilitate technology 
transfer and collaborative R\&D efforts.

\section{Sustainable Development}

Focus on sustainable development to address environmental concerns 
and promote clean technology, reducing pollution and ensuring 
long-term environmental sustainability.

\section{Support for Startups and Entrepreneurship}

Foster a supportive ecosystem for startups and entrepreneurs, 
including access to funding, mentorship, and regulatory 
simplifications, to stimulate innovation and economic growth.

\section{Investment in Critical Technologies}

Prioritize investments in key technologies such as artificial 
intelligence, 5G, and biotechnology to maintain a competitive 
edge in these critical sectors.

\section{Trade and Global Engagement}

Continue active participation in global trade and international 
organizations, fostering cooperation and ensuring open markets 
for Chinese goods and services.

\section{Infrastructure Development}

Invest in further infrastructure development, especially in 
underdeveloped regions, to reduce regional disparities and 
support economic growth.

\section{Cybersecurity and Data Protection}

Strengthen cybersecurity measures and data protection regulations 
to ensure the security of critical infrastructure, technologies, 
and sensitive data.

\section{Economic Resilience Planning}

Create contingency plans and strategies for economic resilience 
in the face of potential global economic challenges or crises.

\section{Summary about Policy and Recommendation}

These policy recommendations and considerations are intended to 
guide the Chinese government, businesses, and other stakeholders 
in continuing to drive technological advancements and sustainable 
economic growth in China while addressing challenges and ensuring 
equitable benefits for the population.
\ifproject
\chapter{Conclusions and Discussions}

\section{Conclusions}

In conclusion, China's remarkable journey of technology and economic growth 
is a testament to the country's resilience, adaptability, and strategic vision. 
Over the past few decades, China has evolved from a primarily agrarian society 
into a global economic powerhouse with a significant influence on the 
technological landscape. Several key points emerge from this exploration

\subsection{Policy-Driven Transformation}

China's economic and technological growth is largely policy-driven. Government 
initiatives, economic reforms, and investment strategies have played a pivotal 
role in propelling China forward.

\subsection{Global Impact}

China's growth has global implications. Its integration into the global economy, 
as well as its innovations in technology, trade, and infrastructure, affect 
nations and industries around the world.

\subsection{Innovation and Technological Advancements}

China's investments in research and development, intellectual property protection, 
and innovation ecosystems have fueled advancements in sectors like 
telecommunications, artificial intelligence, and renewable energy.

\section{Challenges}

While China's growth offers numerous opportunities, it also presents challenges, such 
as environmental sustainability, income inequality, and geopolitical tensions. 
Managing these challenges is essential for long-term stability.

\section{Suggestions and further improvements}

Here are some suggestions for further improvement and considerations regarding China's 
technology and economic growth:

\subsection{Sustainable Development}

Emphasize sustainable development by prioritizing environmentally friendly technologies 
and practices. This includes reducing air and water pollution, conserving resources, 
and promoting renewable energy sources.

\subsection{Innovation Ecosystem}

Continue to cultivate a vibrant innovation ecosystem by supporting startups, providing 
access to venture capital, and fostering a culture of entrepreneurship.

\subsection{Rural Development}

Address regional disparities by promoting economic development and technological 
advancement in less-developed rural areas, reducing the urban-rural economic divide.
\fi

\bibliography{myReport}

\ifproject
\normalspacing
\appendix
\include{chapters/appendix}

%% Display glossary (optional) -- need glossary option.
\ifglossary\glossarypage\fi

%% Display index (optional) -- need idx option.
\ifindex\indexpage\fi

\begin{biosketch}
\begin{center}
  \includegraphics[width=1.5in]{mugshot.jpg}
\end{center}
Your biosketch goes here. Make sure it sits inside
the \texttt{biosketch} environment.
\end{biosketch}
\fi % \ifproject
\end{document}
